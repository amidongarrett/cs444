\documentclass[a4paper]{article}

\usepackage[english]{babel}
\usepackage[utf8x]{inputenc}
\usepackage{amsmath}
\usepackage{graphicx}
\usepackage{bibentry}

\title{Week 4 Summary}
\author{Garrett Amidon}

\begin{document}
\maketitle




\paragraph{}

Chapters 6 and 7 of “Linux Kernel Development 3rd edition” (published June 2010), by Robert Love discussed the different types of data structures in the linux kernel and what they are and how they are used and the complexity of an algorithm followed by what an interrupt is and how an interrupt handler is implemented and what it does. By going in this order of defining a data structure and how it is used, it gets the reader thinking about code, rather than abstract thoughts about a data structure, so when the author gets in to signals and interrupt handling, the reader can follow easier because there are no abstract thoughts to distract the reader. In order for the author to get the reader to understand the material, he uses code examples and builds up from basics, like in chapter 6 he starts with a linked list before going into a circular or double linked list, this method avoids confusion. The audience this section is most shaped towards are people who understand code, such-as in chapter 6 where he defines what a linked list is using C, and follows by showing code example for interrupts and interrupt handlers.






\end{document}
