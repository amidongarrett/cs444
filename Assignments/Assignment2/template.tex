\documentclass[letterpaper,10pt,titlepage,draftclsnofoot,onecolumn]{IEEEtran}
\linespread{1}
\usepackage{graphicx}                                        
\usepackage{amssymb}                                         
\usepackage{amsmath}                                         
\usepackage{amsthm}                                          

\usepackage{alltt}                                           
\usepackage{float}
\usepackage{color}
\usepackage{url}
\usepackage{listings}

\usepackage{balance}
\usepackage[TABBOTCAP, tight]{subfigure}
\usepackage{enumitem}
\usepackage{pstricks, pst-node}

\usepackage{geometry}
\usepackage{titling}
\geometry{textheight=8.5in, textwidth=6in}

%random comment

\newcommand{\cred}[1]{{\color{red}#1}}
\newcommand{\cblue}[1]{{\color{blue}#1}}

\usepackage{hyperref}
\usepackage{geometry}

\def\name{Garrett Amidon}

%pull in the necessary preamble matter for pygments output
\input{pygments.tex}

%% The following metadata will show up in the PDF properties
\hypersetup{
  colorlinks = true,
  urlcolor = black,
  pdfauthor = {\name},
  pdfkeywords = {cs444 ''operating systems 2'' files filesystem I/O},
  pdftitle = {CS 444 Project 1: Getting Acquainted},
  pdfsubject = {CS 444 Project 1},
  pdfpagemode = UseNone
}

\title{CS444: Assignment 2 \\
	\large Spring 2016}
\author{Garrett Amidon}


\begin{document}
\begin{titlingpage}
    \maketitle
	\centering{}
    \begin{abstract}
        This assignment was about building and developing a kernel environment as well as working on changing the default scheduler to be issf instead of noop.   
    \end{abstract}
\end{titlingpage}

\section{Implementation}

For my implementation I went to wikipedia for a general idea of what the sstf algorithm should look like. \cite{wiki} From there I learned that the scheduler maintains an incoming buffer of requests and it determines which request is closest to the current position of the head and services it next. I have a general idea of what the algorithm does but I still have no idea how to implement it. 

\section{Questions}
\subsection{What do you think the main point of this assignment is?}
The main point of this assignment I believe was to test my ability to code an I/O scheduler and changing my default I/O scheduler in the kernel. It was also intended for me to make my first patch to the kernel. 
\subsection{How did you personally approach the problem? Design decisions, algorithm, etc.}
To start, I looked up the Noop algorithm in our block directory. From there I realized I had no idea what was going on so I tried to look up examples of other people's code. After finding plenty of examples and trying for countless hours to try to create my own algorithm, I did not reach a conclusive result. So instead, I have copied the Noop algorithm and add a print statement to see if I can at least switch the scheduler correctly. 
\subsection{How did you ensure your solution was correct?}
As mentioned above, my algorithm isn't correct. I had to decide to cut my loses where applicable and try to get what I believe is the main point of the assignment and that is to create a patch and change the scheduler. 
\subsection{What did you learn?}
I learned that I need to allocate more time to the assignment, although I allocated well over 30 hours. This is one of those things that I just cant seem to understand and I am paying the price. 



\section{Version Control Log}

\begin{tabular}{|| c c c c ||}
\hline
Date & Additions & Deletions & Message\\
\hline
4/20/2016 & 27 & 0 & Update makefile\\
\hline
4/20/2016 & 77 & 0 & Update Kconfig.iosched\\
\hline
4/20/2016 & 157 & 0 & Create sstf-iosched.c\\
\hline
4/27/2016 & 48 & 0 & Final Commit\\

\hline 
\end{tabular}





\section{Work Log}
\begin{itemize}
\item Wednesday 4/20 - Started working on exploring the internet to see what algorithms exist already to see if I can wrap my head around the assignment. I have also tried to understand the noop algorithm. 4 hours. 
\item Wednesday 4/20 - Started working on getting the kernel to change the i/o scheduler and have made changes to the kconfig.iosched and makefile. 5 hours
\item Friday 4/22 - I am trying to get the assignment done by Monday but it is not looking well. From testing I have figured out that my code will compile and the the sstf is set to my default, but when making the kernel it is still making noop the default. I have no clue what to do. 6 hours 
\item Monday 4/25 - I finally got the scheduler to default to issf. I don't know what changed but it's progress. 3 hours.
\item Weds 4/27 - I have no clue how to do my algorithm so I ended up copying the noop algorithm and putting a print statement in it to show that I did in fact change the scheduler. 7 hours

Total time: 25 hours


\end{itemize}

\bibliographystyle{IEEEtran}
\bibliography{template}
\end{document}
